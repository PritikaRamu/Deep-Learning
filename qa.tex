\documentclass[twocolumn]{article}  
\usepackage[top = 1.5cm, bottom = 1.8cm, left = 1.0cm, right = 1.0cm]{geometry}  
\usepackage{graphicx}
\usepackage{amsmath}  
\usepackage{tfrupee} 

% use $ $ to enter and exit math mode

\begin{document}  

\title{DL Assignment - 01}
\author{\bf Pritika Ramu 2019A7PS1140P}

\maketitle

\section{[Page 132-133]}

\begin{enumerate}
        
        \item There is a $10\%$ loss if an article is sold at \rupee270. Then the cost price of the article is \hfill {\bf[SSC 2015]}
            \begin{enumerate}
                \item \rupee250
                \item \rupee270
                \item \rupee300
                \item \rupee320
            \end{enumerate}
        
        
        \item By selling an article for \rupee450. I lose $20\%$. For what price should I sell it to gain $20\%$? \hfill {\bf[SSC 2015]}
            \begin{enumerate}
                \item \rupee470
                \item \rupee490
                \item \rupee562.50
                \item \rupee675
            \end{enumerate}

        
        \item A retailer buys a radio for \rupee225. His overhead expenses are \rupee15. He sells the radio for \rupee300. The profit per cent of the retailer is \hfill {\bf[SSC 2015]}
            \begin{enumerate}
                \item 20 \%
                \item 25 \%
                \item $26 \frac{1}{7} \%$
                \item $33 \frac{1}{3} \%$
            \end{enumerate}
        
        \item The cost price of a radio is \rupee600 . The $5\%$ of the cost price is charged towards transportation. After adding that, if the net profit to be made is $15\%$, then the selling price of the radio must be \hfill {\bf[SSC MTS 2013]}
            \begin{enumerate}
                \item \rupee664.50
                \item \rupee684.50
                \item \rupee704.50
                \item \rupee724.50
            \end{enumerate}
            
        \item If bananas are bought at the rate of 4 for a rupee, how many must be sold for a rupee so as to gain $33 \frac{1}{3} \%$? \\ \strut\hfill {\bf[SSC 2015]}
            \begin{enumerate}
                \item 2
                \item 2.5
                \item 3
                \item 4
            \end{enumerate}
            
        \item A merchant loses $10 \%$ by selling an article. If the cost price of the article is \rupee15, then the selling price of the article is \\ \strut\hfill {\bf[SSC 2014]}
            \begin{enumerate}
                \item \rupee12.30
                \item \rupee13.20
                \item \rupee13.50
                \item \rupee16.50
            \end{enumerate}
            
        \item There is a profit of $20 \%$ on the price of an article. The $\%$ of profit, when calculated on selling price is \\ \strut\hfill {\bf[SSC CGL 2014]}
            \begin{enumerate}
                \item $16 \frac{2}{3} \%$
                \item $20 \%$
                \item $33 \frac{1}{3} \%$
                \item None of these
            \end{enumerate}
            
        \item If selling price of an article is $1 \frac{1}{3}$ of cost price, find the gain \%. \hfill {\bf[SSC 2014]}
            \begin{enumerate}
                \item$1.33 \%$
                \item$25 \%$
                \item$33 \frac{1}{3} \%$
                \item$66 \frac{2}{3} \%$
            \end{enumerate}
        
        \item If books bought at price from \rupee150 to 300 are sold at prices ranging from \rupee250 to \rupee350, what is the greatest possible that might be made in selling 15 books? \hfill {\bf[SSC 2013]}
            \begin{enumerate}
                \item \rupee750
                \item \rupee3000
                \item \rupee4250
                \item Cannot be determined
            \end{enumerate}
            
        \item A fruit merchant makes a profit of $25 \%$ by selling mangoes at a certain price. If he charges \rupee1 more on each mango, he would gain $50 \%$. At first the price of one mango was \\ \strut\hfill {\bf[SSC CPO 2015]}
            \begin{enumerate}
                \item \rupee4
                \item \rupee5
                \item \rupee6
                \item \rupee7
            \end{enumerate}
            
        \item Mahima bought a number of oranges at 2 for a rupee and an equal number at 3 for a rupee. To make a profit of $20 \%$ she should sell a dozen for \hfill {\bf[SSC CGL 2014]}
            \begin{enumerate}
                \item \rupee6
                \item \rupee8
                \item \rupee10
                \item \rupee12
            \end{enumerate}
        
        \item Ravi bought a camera and paid $20 \%$ less than its original price. He sold it at $40 \%$ profit on the price he had paid. The percentage of profit earned by Ravi on the original price was
            \begin{enumerate}
                \item $12 \%$
                \item $15 \%$
                \item $22 \%$
                \item $32 \%$
            \end{enumerate}
            
        \item By selling an article, a man makes a profit of $25 \%$ of its selling price. His profit per cent is \hfill {\bf[SSC CGL 2010]}
            \begin{enumerate}
                \item $16 \frac{2}{3} \%$
                \item $20 \%$
                \item $25 \%$
                \item $33 \frac{1}{3} \%$
            \end{enumerate}
            
        \item If the cost price of 25 chairs is equal to the selling price of 30 chairs, then the loss $\%$ is \hfill {\bf[SSC 2015]}
            \begin{enumerate}
                \item $5 \%$
                \item $16 \frac{2}{3} \%$
                \item $20 \%$
                \item $25 \%$
            \end{enumerate}
            
        \item Kabir buys one kilogram of apples for \rupee120 and sells it to Shashi gaining $25 \%$. Shashi sells it to Geeta who again sells it for \rupee198, making a profit of $10 \%$. What is the profit percentage made by Shashi? \hfill {\bf[SSC SI 2016]}
            \begin{enumerate}
                \item $10 \%$
                \item $20 \%$
                \item $22 \%$
                \item $25 \%$
            \end{enumerate}
            
        \item Ten articles were bought for \rupee8, and sold at 8 for \rupee10. The gain per cent is \hfill {\bf[SSC CGL 2015]}
            \begin{enumerate}
                \item $54.75 \%$
                \item $55 \%$
                \item $56.25 \%$
                \item $57.25 \%$
            \end{enumerate}
            
        \item A man purchased an article for \rupee1500 and sold it at $25 \%$ above the cost price. If he has to pay \rupee75 as tax on it, his net profit percentage will be \hfill {\bf[SSC 2015]}
            \begin{enumerate}
                \item \rupee12.30
                \item \rupee13.20
                \item \rupee13.50
                \item \rupee16.50
            \end{enumerate}
            
        \item A man purchases some oranges at the rate of 3 for \rupee40 and the same quantity at 5 for \rupee60. If he sells all the oranges at the rate of 3 for \rupee50, find his gain or loss per cent (to the nearest integer). \hfill {\bf[SSC CGL 2015]}
            \begin{enumerate}
                \item $32 \%$ profit
                \item $31 \%$ loss
                \item $34 \%$ loss
                \item $31 \%$ profit
            \end{enumerate}
            
        \item Divya purchased $2 \frac{1}{2}$ dozen eggs at the rate of \rupee20 per dozen. She found that 6 eggs were rotten. She sold the remaining eggs at the rate of \rupee22 per dozen. Then her profit or loss per cent is \hfill {\bf[SSC 2013]}
            \begin{enumerate}
                \item $12 \%$ loss
                \item $12 \%$ profit
                \item $10 \%$ loss
                \item $10 \%$ profit
            \end{enumerate}
            
        \item A man bought 20 dozen eggs for \rupee720. What should be the selling price of each egg if he wants to make a profit of $20 \%$? hfill {\bf[SSC 2010]}
            \begin{enumerate}
                \item \rupee3.25
                \item \rupee3.30
                \item \rupee3.50
                \item \rupee3.60
            \end{enumerate}
            
        \item A clock was sold for \rupee144. If the percentage of profit was numerically equal to the cost price, the cost of the clock was
            \begin{enumerate}
                \item \rupee72
                \item \rupee80
                \item \rupee90
                \item \rupee100
            \end{enumerate}
            
        \item A vendor loses the selling price of 4 oranges on selling 36 oranges. His loss per cent is
            \begin{enumerate}
                \item $9 \%$
                \item $10 \%$
                \item $11 \frac{1}{2} \%$
                \item $12 \frac{1}{2} \%_{0}$
            \end{enumerate}
            
        \item If the profit on selling an article for \rupee425 is the same as the loss on selling it for \rupee355, then the cost price of the article is
            \begin{enumerate}
                \item \rupee370
                \item \rupee380
                \item \rupee390
                \item \rupee400
            \end{enumerate}
            
        \item A fruit seller buys 240 apples for \rupee600. Some of these apples are bad and are thrown away. He sells the remaining apples at \rupee3.50 each and makes a profit of \rupee198. The per cent of apples thrown away are \hfill {\bf[SSC 2015]}
            \begin{enumerate}
                \item $5 \%$
                \item $6 \%$
                \item $7 \%$
                \item $8 \%$
            \end{enumerate}
            
        \item $A$ sold an article to $B$ at $20 \%$ profit and $B$ sold to $C$ at $15 \%$ loss. If $A$ sold it to $C$ at the selling price of $B$, then $A$ would make \hfill {\bf[SSC CGL 2014]}
            \begin{enumerate}
                \item $2 \%$ profit
                \item $2 \%$ loss
                \item $5 \%$ profit
                \item $5 \%$ loss
            \end{enumerate}
            
        \item $A$ sells an article to $B$ at a gain of $20 \%$ and $B$ sells it to $C$ at a gain of $10 \%$ and $C$ sells it to $D$ at a gain of $12 \frac{1}{2} \%$. If $D$ pays \rupee29.70, then $A$ purchased the article for \\ \strut\hfill {\bf[SSC 2013]}
            \begin{enumerate}
                \item \rupee10
                \item \rupee20
                \item \rupee30
                \item \rupee40
            \end{enumerate}
            
        \item $A$ sells an article to $B$ at a gain of $10 \%$. $B$ sells it to $C$ at a gain of $7 \frac{1}{2} \%$. $C$ disposes of it at a loss of $25 \%$. If the prime cost to the manufacturer $A$ was \rupee3200 then the price obtained by $C$ is \hfill {\bf[SSC 2013]}
            \begin{enumerate}
                \item \rupee2580
                \item \rupee2670
                \item \rupee2800
                \item \rupee2838
            \end{enumerate}
            
        \item A car worth \rupee150000 was sold by $X$ to $Y$ at $5 \%$ profit sold the car back to $X$ at $2 \%$ loss. In the entire transaction
            \begin{enumerate}
                \item $X$ gained \rupee4350
                \item $Y$ lost \rupee4350
                \item $X$ gained \rupee3150
                \item $X$ lost \rupee3150
            \end{enumerate}
            
        \item Rashi wants to sell a watch at a profit of $20 \%$. She bought it at $10 \%$ less and sold it at \rupee30 less, but she gained $20 \%$. The cost price of watch is \hfill {\bf[SSC CGL 2015]}
            \begin{enumerate}
                \item \rupee220
                \item \rupee225
                \item \rupee240
                \item \rupee250
            \end{enumerate}
            
        \item A dealer sold a bicycle at a profit of $10 \%$. Had he bought the bicycle at $10 \%$ less price and sold it at a price \rupee60 more, he would have gained $25 \%$. The cost price of the bicycle was \hfill {\bf[SSC CGL 2015]}
            \begin{enumerate}
                \item \rupee2000
                \item \rupee2200
                \item \rupee2400
                \item \rupee2600
            \end{enumerate}
            
        \item An article was sold at a profit of $12 \%$. If the cost price would be $10 \%$ less and selling price would be \rupee5.75 more, there would be a profit of $30 \%$. Then at what price it should be sold to make a profit of $20 \% $? \hfill {\bf[SSC 2014]}
            \begin{enumerate}
                \item \rupee115
                \item \rupee120
                \item \rupee138
                \item \rupee215
            \end{enumerate}
            
        \item If the ratio of cost price to selling price is $10: 11$, then the rate of per cent of profit is \hfill {\bf[SSC CGL 2015]}
            \begin{enumerate}
                \item $0.1 \%$
                \item $1 \%$
                \item $1.1 \%$
                \item $10 \%$
            \end{enumerate}
        
        \item A profit of $12 \%$ is made when a mobile phone is sold at \rupee$P$ and there is $4 \%$ loss when the phone is sold at \rupee$Q$. Then $Q: P$ is \hfill {\bf[SSC CGL 2015]}
            \begin{enumerate}
                \item $1: 1$
                \item $4: 5$
                \item $6: 7$
                \item $3: 1$
            \end{enumerate}

        \item The cost price : selling price of an article is $a: b$, if $b$ is $200 \%$ of $a$ then the percentage of profit on cost price is
            \begin{enumerate}
                \item $75 \%$
                \item $100 \%$
                \item $125 \%$
                \item $200 \%$
            \end{enumerate}
        
        \item Veer buys a watch at $\frac{4}{5}$th of its marked price and sells it for $17 \%$ more than its marked price. His profit\% is  \\ \strut\hfill {\bf[SSC CGL 2015]}
            \begin{enumerate}
                \item $37.5 \%$
                \item $40.25 \%$
                \item $46.25 \%$
                \item $50.5 \%$
            \end{enumerate}
        
        \item By selling an article at $\frac{2}{3}$ of the marked price, there is a loss of $10 \%$. The profit per cent, when the article is sold at the marked price, is
            \begin{enumerate}
                \item $20 \%$
                \item $30 \%$
                \item $35 \%$
                \item $40 \%$
            \end{enumerate}
            
        \item The marked price of an article is $50 \%$ above cost price. When marked price is increased by $20 \%$ and selling price is increased by $20 \%$, the profit doubles. If the original marked price is \rupee300, then original selling price is
            \begin{enumerate}
                \item \rupee200
                \item \rupee240
                \item \rupee250
                \item \rupee275
            \end{enumerate}
            
        \item A shopkeeper sold his goods at half the list price and thus lost $20 \%$. If he had sold on the listed price, his gain percentage would be
            \begin{enumerate}
                \item $20 \%$
                \item $35 \%$
                \item $72 \%$
                \item $60 \%$
            \end{enumerate}
\end{enumerate}


\end{document}
